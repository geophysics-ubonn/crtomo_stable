\documentclass[12pt,a4paper,twoside,titlepage,draft]{book}
\usepackage[T1]{fontenc}
\usepackage[utf8]{inputenc}
\usepackage{ngerman}
\usepackage{array}
\usepackage{verbatim}
\usepackage{moreverb}
\usepackage{amsfonts}
\usepackage{amsmath}
\usepackage{amssymb}
\usepackage{multicol}
\usepackage{wasysym}
\usepackage{amsthm}
\usepackage{units}
\usepackage{setspace}
\usepackage{tabularx}
\usepackage{curves}
\usepackage{subfigure}
%\usepackage{wrapfig}
\usepackage[font=small,labelfont=bf]{caption}
\usepackage{textcomp}
\usepackage{eurosym}
\usepackage{gensymb}
\usepackage[pdftex]{graphicx, color}
\usepackage[pdftex,colorlinks=true,urlcolor=blue,linkcolor=blue]{hyperref}
\usepackage{makeidx}
\makeindex
\makeatletter

\newenvironment{tablehere}
  {\def\@captype{table}}
  {}
\newenvironment{figurehere}
  {\def\@captype{figure}}
  {}
\makeatother
%\setlength{\arraycolsep}{4 pt} % 3pt gut
\setlength{\columnseprule}{1pt}

\newcommand{\K}{\mathbb{K}}
\newcommand{\R}{\mathbb{R}}
%\renewcommand{\C}{\mathbb{C}}
\newcommand{\N}{\mathbb{N}}
\newcommand{\Z}{\mathbb{Z}}
\newcommand{\Q}{\mathbb{Q}}
\renewcommand{\P}{\mathbb{P}}
\newcommand{\A}{\mathcal{A}}
\newcommand{\B}{\mathcal{B}}
\newcommand{\del}{\partial}
\DeclareMathOperator{\sign}{sign}
\DeclareMathOperator{\dist}{dist}
\DeclareMathOperator{\diam}{diam}
\DeclareMathOperator{\dom}{dom}
\DeclareMathOperator{\im}{im}
\DeclareMathOperator{\coker}{coker}
\DeclareMathOperator{\codim}{codim}
\DeclareMathOperator{\Mat}{Mat}
\DeclareMathOperator{\ind}{ind}
\DeclareMathOperator{\conv}{conv}
\DeclareMathOperator{\supp}{supp}
\DeclareMathOperator{\e}{e}
\DeclareMathOperator{\Id}{Id}
\DeclareMathOperator{\real}{Re}
\DeclareMathOperator{\imag}{Im}
\DeclareMathOperator{\grad}{grad}
\DeclareMathOperator{\GL}{GL}
\DeclareMathOperator{\erf}{erf}
\providecommand{\norm}[1]{\left\lVert#1\right\rVert}
\providecommand{\abs}[1]{\left\lvert#1\right\rvert}
\newcommand{\gdw}{\Leftrightarrow}
\newcommand{\Gdw}{\Longleftrightarrow}

\newcommand{\crm}{{ CRMod }}
\newcommand{\crt}{{ CRTomo }}
\newcommand{\gri}{{ Griev }}

\newtheoremstyle{aufgaben}%Name
{3pt} % Abstand nach oben
{3pt} % Abstand nach unten
{} % Font der ``section''
{} % Einrücken
{\bf}% Überschrift Font
{:}% Punkt nach der Überschrift
{.5em}% Abstand zwischen Überschrift und Punkt
{}%Überschirft Spezifikation

\theoremstyle{aufgaben}
\newtheorem{aufgabe}{Aufgabe}
\theoremstyle{remark}
\newtheorem{loesung}{Lösung}

\newcommand{\vct}[1]{\mathbf{#1}}
\newcommand{\mf}[1]{\mathsf{#1}}

\setlength{\parindent}{0pt}
\usepackage{geometry}
\geometry{top=4cm,head=2cm,headsep=1cm,bottom=2cm,right=1cm,left=1cm}
%\usepackage{ccfonts}
%\usepackage{eulervm}



\usepackage{datenumber}
\newcommand{\pnext}{%
\ifnum\value{dateday}<10 0\fi
\thedateday.%
\ifnum\value{datemonth}<10 0\fi
\thedatemonth.\thedateyear
\nextdate
}
\usepackage{fancyhdr}

%% L/C/R denote left/center/right header (or footer) elements
%% E/O denote even/odd pages

%% \leftmark, \rightmark are chapter/section headings generated by the 
%% book document class

%opening
\title{Operation Manual \\[.1cm] Complex Resistivity suite \\[.3cm] (CRMod/CRTomo) }
\author{\textcopyright Andreas Kemna}
\newcommand{\mydate}{April 2010}


\fancyhead{}
\fancyhead[LE,RO]{\slshape \thepage}
\fancyhead[RE]{\slshape \leftmark}
\fancyhead[LO]{\slshape \rightmark}
\fancyfoot{}
\fancyfoot[LO,LE]{\slshape Operation Manual for CRMod/CRTomo }
\fancyfoot[RO,RE]{\slshape \mydate}

\begin{document}
\pagestyle{fancy}

\selectlanguage{english}

\pagenumbering{roman}

\maketitle
\thispagestyle{empty}
% \vfill
% \begin{center}
% {\large
% {\huge
% \textcopyright by Andreas Kemna \\[1cm] \mydate}
% \end{center}
% \vfill
\cleardoublepage
\pagenumbering{arabic}
\setcounter{page}{1}

\addcontentsline{toc}{chapter}{Table of contents}
\tableofcontents
\addcontentsline{toc}{chapter}{List of figures}
\listoffigures
\addcontentsline{toc}{chapter}{List of tables}
\listoftables

\chapter{Introduction} \label{chap:intro}
\section{General remarks}\label{sect:intro_general}

\subsection{\crm}\label{sub:intro_crm}
The forward modelling tool \crm \index{\crm} is a finite-element-based program for 2.5D modelling in electrically conductive and polarizable media.
It calculates the electric potential due to a low-frequency (pseudo-dc) electric current point-source in a cross-scetion perpendicular to the strike direction of a two-dimensionallly heterogeneous medium.
Since the potential is integrated along the direction of the strike in the wavenumber-frequency domain, it is a 2.5D approximation.
The medium, typically represents the Earth's subsurface, but may also represent other objects as confined tanks or vessels.
The finite-element method grants a huge range of flexibility in meshing the underlying medium.

For any given two-dimensional complex resistivity (comprising magnitude and phase) distribution, the modelled response is either a set of complex potential distributions in the considered cross-sectional planefor a given set of one- and/or two-pole current injection confugurations; a set of impedance values for a given set of two-, three- and/or four-pole measurement confugurations (one or two poles for current injection, another one or two poles for voltages measurement); and/or a set of complex sensitivity distributions in the cross-sectional plane corresponding to the given set of measurement configurations.
If polarizability of the medium is disregarded, the underlying resistivity distribution, as well as the modelled potential


\subsection{\crt}\label{sub:intro_crt}
The inverse modelling tool \crt \index{\crt} is the corresponding tomographic program to \crm.


%%%%%%%%%%%%%%%%%%%%%%%%%%%%%%%%%%%%5
\chapter{\crm}\label{chap:crmod}
\section{Boundary value problem and FE-Method}\label{sect:crmod_boundary_fe}
\subsection{Dirichlet}\label{sub:crmod_dirichlet}
\subsection{Neumann}\label{sub:crmod_neumann}
\subsection{Mixed boundary values}\label{sub:crmod_mixed}
\subsection{2D vs. 2.5D}\label{sub:crmod_2dvs25d}
\clearpage
\section{Induced Polarization (IP)}\label{sub:crmod_ip}
\subsection{Spectral IP}\label{sub:crmod_sip}

\clearpage
\section{Folder structure and input files}\label{sect:crmod_dir}
\subsection{crmod.cfg}\label{sub:crmod_input}
\subsection{Grid file (elem.dat)}\label{sub:crmod_grid}
\subsection{Electrode file (elec.dat)}\label{sub:crmod_elec}
\subsection{Configuration file (conf.dat)}\label{sub:crmod_conf}
\subsection{Model file (model.dat)}\label{sub:crmod_modl}
\subsection{Pseudo data file (volt.dat)}\label{sub:crmod_volt}

%%%%%%%%%%%%%%%%%%%%%%%%%%%%%%%%%%%%%%%%%
\chapter{\crt}\label{chap:crtomo}
\section{Electrical Resitivity Tomography (ERT)}\label{sect:crtomo_ert}
\subsection{Ordinary least squares (OLS) / Gauss-Newton optimization}\label{sect:crtomo_ert}
\subsection{Adjoint boundary value problem and Sensitivities}\label{sub:crtomo_adj}
\subsection{Conjugate Gradient linear solver}\label{sect:crtomo_ert}

\clearpage
\section{Electrical Impedance Tomography (EIT)}\label{sect:crtomo_eit}
\subsection{Complex Sensitivities}\label{sub:crtomo_eit_adj}
\subsection{Final Phase Improvement (FPI)}\label{sub:crtomo_eit_fpi}

\clearpage
\section{Regularization\index{Regularization}}\label{sect:crtomo_reg}
\subsection{Smoothness\index{Smoothness} constraints}\label{sub:crtomo_reg_smooth}
\subsection{Levenberg and Levenberg-Marquardt damping}\label{sub:crtomo_reg_lma}
\subsection{Minimum Support and Minimum Gradient Support}\label{sub:crtomo_reg_mgs}

\clearpage
\section{Stochastic (regularization) / Geostatistics}\label{sect:crtomo_sto}
\subsection{Empirical Variogram and Covariance estimator}\label{sub:crtomo_sto_empvar}
\subsection{Variogram and Covariance models}\label{sub:crtomo_sto_var}

\clearpage
\section{Error Model and Noise}\label{sct:cortomo_errmod}
\subsection{Data Error Model}\label{sct:cortomo_data_errmod}
\subsection{Noise implementation}\label{sct:cortomo_noise_errmod}

\clearpage
\section{Folder structure and input files}\label{sect:crtomo_dir}
\subsection{crtomo.cfg}\label{sub:crtomo_input}
\subsection{Grid file (elem.dat)}\label{sub:crtomo_grid}
\subsection{Electrode file (elec.dat)}\label{sub:crtomo_elec}
\subsection{Pseudo data file (volt.dat)}\label{sub:crtomo_volt}
\subsection{Prior data}\label{sub:crtomo_prior}
\subsection{Difference inversion}\label{sub:crtomo_diffinv}

%%%%%%%%%%%%%%%%%%%%%%%%%%%%%%%%%%%%%%%%%
\printindex
\end{document}
